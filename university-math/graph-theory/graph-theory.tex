\documentclass[../../mathematics_cheat_sheet.tex]{subfiles}

\begin{document}

\chapter{Graph Theory}
This chapter will be based on~\citet{GodsilRoyle2001}\cite{GodsilRoyle2001}.
\section{Fundamentals}
\subsection{Vertex:}
The fundamental unit of \emph{graph theory}.
\[v\]
\subsection{Edge:}
An unordered pair of vertices.
\[e=\{v_{1},v_{2}\}\]
\subsection{Vertex Set:}
\[
V=\left\{v_{1}, v_{2}, v_{3}, \ldots\right\}
\]
\subsection{Edge set:}
\[
E=\left\{e_{1}, e_{2}, e_{3}, \ldots\right\}
\]
\subsection{Graph:}
A Graph $X$ is composed of a \emph{vertex} set $V(X)$ and a \emph{edge} set $E(X)$.
\subsection{Subgraphs:}
\[
\begin{array}{cc}
\text{Any subgraph } \mathrm{H} \text{ such that}\\
V(H) \subset V(G) ~\&~ E(H) \subset E(G)
\end{array}
\]
\subsection{Tree:}
\begin{center}
Any subgraph $\mathrm{H}$ where $V(H)=V(G)$,\\ there are no cycles and all vertices are connected.
\end{center}
%
\subsection{Degree of vertex:}
\[\text{Number of edges leaving a vertex}\]
\[\sum_{v \in V(G)} d(v)=2|E(G)| \]
%
%
\subsection{Distance:}
\begin{center}
$\quad d(u, v)=$ Shortest path between $u ~\&~ v$
\end{center}
%
\subsection{Diameter:}
\[\quad \text{diam}(G)=\max _{u \& v \in V(G)}\{d(u, v)\}\]
%
\subsection{Total Edges in a simple bipartite graph:}
\[|E(G)|=\frac{|V(X) \| V(Y)|}{2} \]
\[\sum_{x \in X} d(x)=\sum_{y \in Y} d(y)\]
%
\subsection{Total Edges in K-regular graph:}
\[|E(G)|=\frac{k(k-1)}{2}\]
%
\section{Factorisation}
%
\subsection{Factorisation:}
\begin{center}
A spanning union of 1 Factors and only exists\\ if there are an even number of vertices.
\end{center}
%
\subsection{Factors of a $K_\text{n,n}$ bipartite graph:}
\[
\begin{array}{l}
F_{1}=\left[11^{\prime}, 22^{\prime}, 33^{\prime}, \ldots\right]\\
F_{2}=\left[12^{\prime}, 23^{\prime}, 34^{\prime}, \ldots\right]\\
F_{3}=\left[13^{\prime}, 24^{\prime}, 35^{\prime}, \ldots\right]\\
F_{n}=\ldots\\
\text{where all numbers are MOD(n) }
\end{array}
\]
%
\subsection{Factors of a $K_\text{2n}$ graph:}
\[
\begin{array}{l}
F_{0}=\{(1, \infty),(2,0),(3,2 n-2), \ldots,(n, n+1)\}\\
F_{i}=\{(i, \infty),(i+1,2 n-2+1), \ldots,(i+n-1, i+n\}\\
F_{2 n-2}=\ldots\\
\text{where all numbers are } \mathrm{MOD}(2 \mathrm{n}-1)
\end{array}
\]
%
%
\section{Vertex Colouring}
%
\subsection{Chromatic Number:}
\[
\begin{array}{l}
\chi(G) \geq 3 \text{   if there are triangles or an odd cycle}\\
\chi(G) \geq 2 \text{   if is an even cycle}\\
\chi(G) \geq n \text{   if } K_{n} \text{ is a subgraph of G}
\end{array}
\]
%
\subsection{Union/Intersection:}
\[\text { If } G=G_{1} \cup G_{2} \text {  and  } G_{1} \cap G_{2}=K_{m}, \text {  then }\]
\[P(G, \lambda)=\frac{P\left(G_{1}, \lambda\right) P\left(G_{2}, \lambda\right)}{P\left(K_{m}, \lambda\right)}\]
%
\subsection{Edge Contraction:}
% fixed error G.e, see page 5 on https://www.whitman.edu/Documents/Academics/Mathematics/fouts.pdf
\[ P(G, \lambda)=P(G-e, \lambda)-P(G/e, \lambda)\]
%
\subsection{Common Chromatic Polynomials:}
\[
\begin{array}{l}
P\left(T_{n}, \lambda\right)=\lambda(\lambda-1)^{n-1} \\
P\left(C_{n}, \lambda\right)=(\lambda-1)^{n}+(-1)^{n}(\lambda-1) \\
P\left(K_{n}, \lambda\right)=\lambda(\lambda-1)(\lambda-2) \ldots(\lambda-n+1)
\end{array}
\]
%
\begin{center}
 \begin{minipage}{0.75\textwidth}
 \begin{itemize}
 \setlength\itemsep{0em}
\item Where the highest power is the number of vertices
\item Where the lowest power is the number of components
\item Where the coefficient of the 2nd highest power is the number of edges.
\end{itemize}
\end{minipage}
\end{center}
%
\section{Edge Colouring}
%
\subsection{Common Chromatic Polynomials}
\[
\begin{array}{ll}
\chi^{\prime}(G) &\geq \Delta(G) \\
\chi^{\prime}\left(K_{n, n}\right)&=n \\
\chi^{\prime}\left(C_{2 n}\right)&=2 \\
\chi^{\prime}\left(C_{2 n+1}\right)&=3 \\
\chi^{\prime}\left(K_{2 n}\right)&=2 n-1 \\
\chi^{\prime}\left(K_{2 n+1}\right)&=2 n+1
\end{array}
\]

\end{document}
