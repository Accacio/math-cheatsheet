
\chapter{Numbers}

Numbers were created to represent different quantities of things with similar properties: rocks, grains, logs, seeds, cattle etc.
%% TODO:(accacio) add figures to represent different quantities of things (apples, rocks, birds, etc)

Many numbering systems were created by different people, in different contexts, but they all share the counting purpose and have correspondent values.

Here are some examples:
\subsection{Hindu-Arabic numbers}
\[
  {1}\;
  {2}\;
  {3}\;
  {4}\;
  {5}\;
  {6}\;
  {7}\;
  {8}\;
  {9}
\]
\[
  {10}\;
  {20}\;
  {30}\;
  {40}\;
  {50}\;
  {60}\;
  {70}\;
  {80}\;
  {90}\;
  {100}
\]

\subsection{Mayan numbers}
\[
  \maya{1}\;
  \maya{2}\;
  \maya{3}\;
  \maya{4}\;
  \maya{5}\;
  \maya{6}\;
  \maya{7}\;
  \maya{8}\;
  \maya{9}
\]
\[
  \maya{10}\;
  \maya{20}\;
  \maya{30}\;
  \maya{40}\;
  \maya{50}\;
  \maya{60}\;
  \maya{70}\;
  \maya{80}\;
  \maya{90}\;
  \maya{100}
\]

\subsection{Roman numbers}
\[
  \rom{1}\;
  \rom{2}\;
  \rom{3}\;
  \rom{4}\;
  \rom{5}\;
  \rom{6}\;
  \rom{7}\;
  \rom{8}\;
  \rom{9}
\]
\[
  \rom{10}\;
  \rom{20}\;
  \rom{30}\;
  \rom{40}\;
  \rom{50}\;
  \rom{60}\;
  \rom{70}\;
  \rom{80}\;
  \rom{90}\;
  \rom{100}
\]

Great part of the world uses the Hindu-Arabic numeral system.

In~\ref{fig:birds,fig:stones,fig:apples} we can see the correspondence of the symbols with quantities of different objects.

%% TODO(accacio) add figure of birds, stones etc

\chapter{Algebra \& Arithmetic}
\chapter{Geometry}
\section{2D Geometry}
\subsection{Area}
\section{Trigonometry}
\section{3D Geometry}
\subsection{Volume}
\subsection{Surface Area}

%%% Local Variables:
%%% mode: latex
%%% TeX-master: "../mathematics_cheat_sheet"
%%% End:
