\documentclass[../../mathematics_cheat_sheet.tex]{subfiles}
\begin{document}

\chapter{Arithmetic}
The arithmetic, from the greek
$\alpha\rho\iota\theta\mu\eta\tau\iota\kappa\acute\eta$ ($\alpha\rho\iota\theta\mu\eta\tau\iota\mu\acute o\sigma$ \emph{arithmos}, meaning ``number'' and $\tau\iota\kappa\acute\eta$ \emph{tiké}, meaning ``art'') as the name says is the art of using the numbers and studying its uses.

Historically, for the uses of counting 2 principle operations were defined:
\begin{enumerate}
  \item Addition
  \item Subtraction
\end{enumerate}
The idea of addition is to make things together and increase the count and subtraction to remove and decrease the values.
\section{Addition}
For example, if we have 3 apples (\mbox{\apple\apple\apple}) and we want to add 2 other apples (\mbox{\apple\apple}), we represent this by an addition operation, represented by the symbol \emph{$+$}:
\begin{equation*}
  \label{eq:1}
  \begin{array}{ccccc}
    \mbox{\apple\apple\apple}&&\mbox{\apple\apple}&&\mbox{\apple\apple\apple\apple\apple}\\
    &+&&=&\\
    3& &2&&5
  \end{array}
\end{equation*}

\section{Subtraction}
And if we have 7 lemons (\mbox{\lemon\lemon\lemon\lemon\lemon\lemon\lemon}) and we want to remove 3 of them (\mbox{\lemon\lemon\lemon}), we represent this by a subtraction operation, represented by the symbol \emph{$-$}:

\begin{equation*}
  \label{eq:1}
  \begin{array}{ccccc}
    \mbox{\lemon\lemon\lemon\lemon\lemon\lemon\lemon}&&\mbox{\lemon\lemon\lemon}&&\mbox{\lemon\lemon\lemon\lemon}\\
    &-&&=&\\
    7& &3&&4
  \end{array}
\end{equation*}

\end{document}
