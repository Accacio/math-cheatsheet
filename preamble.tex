
\usepackage{amsmath}
\usepackage{tikz}
\usetikzlibrary{graphs,quotes,graphs.standard}
\usepackage{blindtext}
\usepackage{seqsplit}
%% hieroglyphic numbers
\usepackage{hieroglf}
\newcounter{hieroaux}
\newcounter{hierocurr}
\newcommand{\HieroDigit}[1]{\ifcase#1
\or
\Hone
\or
\Hten
\or
\Hhundred
\or
\Hthousand
\or
\HXthousand
\or
\HCthousand
\or
\Hmillion
\fi}
\DeclareRobustCommand\hg[1]{\setcounter{hierocurr}{#1}%
\ifnum\value{hierocurr}>999999
 \setcounter{hieroaux}{0}%
 \loop\stepcounter{hieroaux}\textpmhg{\HieroDigit{7}}%
 \ifnum\value{hieroaux}<\the\numexpr(\value{hierocurr}-500000)/1000000\relax%
 \repeat%
 \setcounter{hierocurr}{\the\numexpr\value{hierocurr}-\value{hieroaux}*1000000}%
\fi
\ifnum\value{hierocurr}>99999
 \setcounter{hieroaux}{0}%
 \loop\stepcounter{hieroaux}\textpmhg{\HieroDigit{6}}%
 \ifnum\value{hieroaux}<\the\numexpr(\value{hierocurr}-50000)/100000\relax%
 \repeat%
 \setcounter{hierocurr}{\the\numexpr\value{hierocurr}-\value{hieroaux}*100000}%
\fi%
\ifnum\value{hierocurr}>9999
 \setcounter{hieroaux}{0}%
 \loop\stepcounter{hieroaux}\textpmhg{\HieroDigit{5}}%
 \ifnum\value{hieroaux}<\the\numexpr(\value{hierocurr}-5000)/10000\relax%
 \repeat%
 \setcounter{hierocurr}{\the\numexpr\value{hierocurr}-\value{hieroaux}*10000}%
\fi%
\ifnum\value{hierocurr}>999
 \setcounter{hieroaux}{0}%
 \loop\stepcounter{hieroaux}\textpmhg{\HieroDigit{4}}%
 \ifnum\value{hieroaux}<\the\numexpr(\value{hierocurr}-500)/1000\relax%
 \repeat%
 \setcounter{hierocurr}{\the\numexpr\value{hierocurr}-\value{hieroaux}*1000}%
\fi%
\ifnum\value{hierocurr}>99
 \setcounter{hieroaux}{0}%
 \loop\stepcounter{hieroaux}\textpmhg{\HieroDigit{3}}%
 \ifnum\value{hieroaux}<\the\numexpr(\value{hierocurr}-50)/100\relax%
 \repeat%
 \setcounter{hierocurr}{\the\numexpr\value{hierocurr}-\value{hieroaux}*100}%
\fi%
\ifnum\value{hierocurr}>9
 \setcounter{hieroaux}{0}%
 \loop\stepcounter{hieroaux}\textpmhg{\HieroDigit{2}}%
 \ifnum\value{hieroaux}<\the\numexpr(\value{hierocurr}-5)/10\relax%
 \repeat%
 \setcounter{hierocurr}{\the\numexpr\value{hierocurr}-\value{hieroaux}*10}%
\fi%
\ifnum\value{hierocurr}>0
 \setcounter{hieroaux}{0}%
 \loop\stepcounter{hieroaux}\textpmhg{\HieroDigit{1}}%
 \ifnum\value{hieroaux}<\value{hierocurr}\relax%
 \repeat%
\fi}

%% maya numbers | put after amsmath
\usepackage{mathabx}
\newcommand\mathbfont{\usefont{U}{mathb}{m}{n}}
\def\mayaexpansion{%
    \mayacntc=\mayacnta\mathbfont
    \ifnum\mayacntc=0 0\else
    \rotatebox[origin=c]{-90}{%
    \loop\ifnum\mayacntc>5\advance\mayacntc by -5\repeat
    \the\mayacntc\mayacntc=\mayacnta
    \loop\ifnum\mayacntc>5\advance\mayacntc by -5 5\repeat}%
  \fi}%
%% roman numbers
\newcommand*{\rom}[1]{\romannumeral #1}

%% drawings
\usetikzlibrary{svg.path}
\def\apple{
\tikz{
  \begin{scope}[scale=-.05]
\draw[fill,red] svg "m63.714 34.589c-4.176-1.764-9.108-2.232-15.084-1.548-6.12 0.684-14.508 1.332-20.592 5.832-6.12 4.428-12.78 11.304-15.408 20.628-2.7 9.36-3.492 22.715-0.288 34.776 3.312 11.951 10.944 28.115 19.692 36.287 8.604 7.885 20.88 10.297 31.68 11.412 10.728 1.117 22.968-0.898 32.292-4.643 9.18-3.961 17.279-10.008 22.788-18.145 5.363-8.279 9.144-19.584 9.54-30.457 0.18-10.906-3.168-25.487-7.704-34.163-4.536-8.712-11.772-13.86-19.08-17.532-7.416-3.708-15.588-5.04-24.912-4.284-0.576-1.872-0.685-3.924-0.324-6.156 0.252-2.304 0.972-4.716 2.16-7.38 0.396-1.62 0.036-2.952-1.224-3.996-1.368-1.188-4.717-2.664-6.156-2.484-1.404 0.144-2.124 1.224-2.16 3.384 0.54 2.844 1.188 5.616 1.548 8.604 0.468 2.952 0.648 5.832 0.937 8.928-2.845-8.712-7.776-15.732-14.796-21.204-7.236-5.544-17.784-10.188-27.36-11.7-9.505-1.574-19.117-0.711-29.233 2.457 5.544 7.452 12.204 13.248 19.98 17.532 7.704 4.212 18.324 6.048 26.46 8.028 8.064 2.016 15.3 3.06 21.852 3.672-1.512 0.72-3.06 1.441-4.608 2.16z"
;
\end{scope}
}
}
\definecolor{lemoncolor}{RGB}{249,199,0}
\def\lemon{
  \raisebox{-2pt}{\tikz{
  \begin{scope}[scale=.05]
    \draw[fill,lemoncolor] svg "m 27.362646,136.35568 c 0,0 0.720069,-6.2406 0.960092,-7.44072 0.240024,-1.20011 4.320418,-5.28051 4.560441,-6.96067 0.240024,-1.68016 3.360325,-3.36032 3.360325,-5.04049 -0.146118,-7.38942 2.402742,-14.34543 3.145853,-21.596377 4.325787,-42.209107 74.839633,-79.304902 119.746033,-55.931115 0,0 5.28051,-1.920186 7.68074,-1.680162 14.39672,-3.436986 38.02336,17.47666 17.28167,36.483527 l -1.44014,0.480046 C 193.5041,148.65244 93.236071,166.57085 54.005221,142.83631 c 0,0 -5.520533,-1.44014 -7.680742,-1.44014 -2.160209,0 -9.840952,0 -9.840952,0 0,0 -3.360324,-0.72007 -4.320417,-1.44014 -0.960093,-0.72007 -4.800464,-3.60035 -4.800464,-3.60035 z";
\end{scope}}}
}



\usepackage{booktabs}
\usepackage{graphicx}
\usepackage{subcaption}
\graphicspath{{img/}}

\newcommand{\monthyear}{%
  \ifcase\month\or January\or February\or March\or April\or May\or June\or
  July\or August\or September\or October\or November\or
  December\fi\space\number\year
}

\usepackage{fancyvrb}
\newcommand{\na}{--}
\newcommand{\blankpage}{\newpage\hbox{}\thispagestyle{empty}\newpage}
\usepackage{units}
\usepackage{makeidx}

\makeindex

%%% Local Variables:
%%% mode: latex
%%% TeX-master: "mathematics_cheat_sheet"
%%% End:
